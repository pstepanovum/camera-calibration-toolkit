\documentclass[conference]{IEEEtran}
% ============================================================================
% PACKAGE IMPORTS
% ============================================================================
\usepackage{amsmath,amssymb,amsfonts}
\usepackage{algorithmic}
\usepackage{graphicx}
\usepackage{textcomp}
\usepackage{xcolor}
\usepackage{multirow}
\usepackage{booktabs}
\usepackage{array}
\usepackage{cite}
\usepackage{url}
\usepackage{enumitem}
\setlist[itemize]{
    label=--,
    leftmargin=2.5em,
    itemsep=1pt,
    parsep=0pt,
    topsep=2pt
}
\setlist[enumerate]{
    leftmargin=*,
    itemsep=1pt,
    parsep=0pt,
    topsep=2pt
}
\usepackage{balance}
\usepackage{hyperref}
\hypersetup{
    colorlinks=true,
    linkcolor=black,
    citecolor=blue,
    urlcolor=blue,
    pdftitle={Feature Detection in Calibrated Stereo Images},
    pdfauthor={Pavel Stepanov}
}
\setlength{\parindent}{0pt}
% ============================================================================
% CUSTOM COMMANDS
% ============================================================================
\newcommand{\mb}[1]{\mathbf{#1}}
\newcommand{\mc}[1]{\mathcal{#1}}
\newcommand{\R}{\mathbb{R}}
% ============================================================================
% SECTION FORMATTING
% ============================================================================
\renewcommand{\thesection}{\arabic{section}}
\renewcommand{\thesubsection}{\thesection.\arabic{subsection}}
% ============================================================================
% DOCUMENT METADATA
% ============================================================================
\def\BibTeX{{\rm B\kern-.05em{\sc i\kern-.025em b}\kern-.08em
    T\kern-.1667em\lower.7ex\hbox{E}\kern-.125emX}}
\pagestyle{plain}
% ============================================================================
% BEGIN DOCUMENT
% ============================================================================
\begin{document}
% ============================================================================
% TITLE AND AUTHORS
% ============================================================================
\title{Feature Detection in Calibrated Rectified Stereo Images:
A Comparative Study of Harris, SIFT, ORB, and KAZE\\
{\normalsize \textit{ECE 738 - Computer Vision, Spring 2026}}}
\author{
    \IEEEauthorblockN{Pavel Stepanov}
    \IEEEauthorblockA{
        \textit{Department of Electrical and Computer Engineering}\\
        \textit{University of Miami}\\
        Coral Gables, Florida, USA\\
        pas273@miami.edu
    }
}
\maketitle
\thispagestyle{plain}
% ============================================================================
% ABSTRACT
% ============================================================================
\begin{abstract}
This paper presents a comparative study of four feature detectors —
Harris, SIFT, ORB, and KAZE — applied to calibrated rectified stereo image
pairs of a laboratory scene. Detectors are evaluated on three stereo pairs
across five criteria: raw feature yield, post-NMS feature count, left-right
stereo correspondence rate, inter-detector overlap, and cross-pair
repeatability. Non-maximum suppression (NMS) is implemented using a fast
grid-based algorithm with suppression radius $D \in \{10, 20\}$ pixels.
Results show that Harris produces the largest number of features (12,288
per image at $D=10$) but saturates the NMS grid, reflecting its
non-selective nature. KAZE achieves the best balance of feature density
(${\sim}2{,}800$ after NMS) and stereo correspondence (${\sim}51\%$), while
also attaining the highest cross-pair repeatability (73--75\% at 15 px
radius), consistent with its nonlinear diffusion scale-space design. SIFT
reaches 67--68\% repeatability, matching published benchmark values from
Mikolajczyk and Schmid. ORB provides the fastest detection but the lowest
stereo correspondence (27--31\%) and repeatability (50--59\%). Detector
overlap analysis reveals that SIFT, ORB, and KAZE are near-perfect subsets
of Harris, while SIFT and KAZE show 62--82\% mutual agreement, confirming
they detect qualitatively similar, physically meaningful features.
\end{abstract}
% ============================================================================
% MAIN CONTENT
% ============================================================================
% ============================================================================
% INTRODUCTION
% ============================================================================
\section{Introduction}

Feature detection is a foundational step in a wide range of computer vision
tasks including 3-D reconstruction, stereo matching, visual localization, and
object recognition. A feature detector identifies salient image locations —
corners, blobs, or texture extrema — that can be reliably re-detected across
different views of the same scene. The quality of subsequent processing
pipelines depends critically on how many such features are found, how well
they are balanced across stereo views, and how reproducibly they are detected
as the camera moves.

This project applies four fundamentally different detectors to calibrated
rectified stereo image pairs of a laboratory scene: Harris \cite{harris1988},
SIFT \cite{lowe2004}, ORB \cite{rublee2011}, and KAZE \cite{alcantarilla2012}.
The images were first calibrated using a checkerboard pattern to obtain
intrinsic parameters and the stereo geometry $(R, T)$, then rectified so that
corresponding points are guaranteed to lie on the same horizontal scanline.
This epipolar constraint is exploited throughout the correspondence analysis.

Three stereo pairs (Pairs 1, 7, and 13) were selected from the available
dataset to represent different scene positions. All four detectors are applied
to both the left and right images of each pair. Results are post-processed
with non-maximum suppression (NMS), evaluated quantitatively, and compared
against published repeatability benchmarks.

% ============================================================================
% BACKGROUND
% ============================================================================
\section{Background and Related Work}

\subsection{Harris Corner Detector}

Harris and Stephens \cite{harris1988} define a corner as a point where image
intensity changes strongly in all directions. Given image gradients $I_x$ and
$I_y$, the structure tensor $\mb{M}$ is computed over a local window:
\begin{equation}
    \mb{M} = \begin{bmatrix}
        \sum I_x^2 & \sum I_x I_y \\
        \sum I_x I_y & \sum I_y^2
    \end{bmatrix}
\end{equation}
The cornerness response is:
\begin{equation}
    R = \det(\mb{M}) - k \cdot \mathrm{tr}(\mb{M})^2
\end{equation}
where $k \approx 0.04$. $R > 0$ indicates a corner, $R < 0$ an edge, and
$R \approx 0$ a flat region. Harris is not scale-invariant but is highly
sensitive and produces a dense response map.

\subsection{SIFT}

Scale-Invariant Feature Transform (SIFT) \cite{lowe2004} detects local
extrema in a Difference-of-Gaussians (DoG) pyramid constructed across
multiple scales $\sigma$:
\begin{equation}
    D(x, y, \sigma) = G(x, y, k\sigma) * I - G(x, y, \sigma) * I
\end{equation}
Points that are extrema in both position and scale are retained after
discarding low-contrast responses (contrast threshold) and edge responses
(ratio of principal curvatures). The detector is invariant to scale and
rotation.

\subsection{ORB}

Oriented FAST and Rotated BRIEF (ORB) \cite{rublee2011} combines the FAST
corner detector with an orientation assignment based on the intensity centroid
of the local patch. FAST tests a ring of 16 pixels around a candidate point:
if $N$ consecutive pixels are all brighter or all darker than the center
by a threshold $t$, a corner is declared. Orientation is assigned as:
\begin{equation}
    \theta = \mathrm{atan2}\!\left(\sum_{p \in P} y \cdot I(p),\;
    \sum_{p \in P} x \cdot I(p)\right)
\end{equation}
ORB is computationally efficient and patent-free, designed as a fast
alternative to SIFT and SURF.

\subsection{KAZE}

KAZE \cite{alcantarilla2012} constructs its scale space through nonlinear
diffusion rather than Gaussian blurring. The diffusion equation:
\begin{equation}
    \frac{\partial L}{\partial t} = \nabla \cdot \left(c(\mb{x},t)\,\nabla L\right)
\end{equation}
uses a conductivity function $c(\mb{x},t)$ derived from the image gradient
magnitude. Near edges, $c \to 0$, halting diffusion; in flat regions,
$c \to 1$, allowing smoothing to proceed. This preserves object boundaries
better than Gaussian blurring, producing features that are more precisely
localised at edges and more stable under viewpoint change.

\subsection{Non-Maximum Suppression}

NMS enforces a minimum spacing $D$ between retained features. The exhaustive
pairwise algorithm is $O(n^2)$, which is intractable for Harris with
$n > 10^6$ responses. We implement a fast grid-based NMS that partitions
the image into $D \times D$ cells, retaining only the highest-response
keypoint per cell. This reduces complexity to $O(n \log n)$ for sorting
plus $O(n)$ for grid assignment, making it practical for any detector.

% ============================================================================
% EXPERIMENTAL SETUP
% ============================================================================
\section{Experimental Setup}

\subsection{Stereo Calibration and Rectification}

Prior to feature detection, the stereo camera system was calibrated using
15 paired checkerboard images (14 successful pairs after one detection
failure). Calibration was performed with \texttt{cv2.stereoCalibrate},
yielding a reprojection error of 1.147 pixels, a baseline of 255.51 mm,
and rotation angles of $(R_x, R_y, R_z) = (-0.52°, -0.92°, -0.67°)$
between cameras. Rectification was computed with $\alpha = 0.8$
(\texttt{cv2.stereoRectify}), producing horizontal epipolar lines and
ensuring that corresponding points share the same image row.

\subsection{Image Pairs and Resolution}

Three stereo pairs were selected: Pair 1 (close frontal view), Pair 7
(intermediate distance), and Pair 13 (wider angle). All images are
$1280 \times 960$ pixels (1.2 megapixels), processed in grayscale.

\subsection{Detector Parameters}

Parameters were tuned away from defaults to maximize strong feature
coverage. Table~\ref{tab:params} summarizes the key settings.

\begin{table}[h]
\centering
\caption{Detector Parameter Settings}
\label{tab:params}
\begin{tabular}{lll}
\toprule
\textbf{Detector} & \textbf{Parameter} & \textbf{Value (Default)} \\
\midrule
Harris & Threshold fraction & 0.10 (N/A) \\
       & Block size / $k$   & 2, 0.04 \\
\midrule
SIFT   & Contrast threshold  & 0.02 (0.04) \\
       & Edge threshold      & 15 (10) \\
\midrule
ORB    & FAST threshold      & 7 (20) \\
       & Max features        & 8000 (500) \\
\midrule
KAZE   & Detector threshold  & 0.0003 (0.001) \\
       & Octaves             & 4 (4) \\
\bottomrule
\end{tabular}
\end{table}

NMS was applied at $D \in \{10, 20\}$ pixels. All analysis reported
in Section~\ref{sec:results} uses $D = 10$ unless stated otherwise.

% ============================================================================
% RESULTS
% ============================================================================
\section{Results and Analysis}
\label{sec:results}

\subsection{Feature Count and Stereo Balance}

Table~\ref{tab:counts} reports raw and post-NMS feature counts and
stereo correspondence rates for all detectors across all three pairs.

\begin{table}[h]
\centering
\caption{Feature Counts and Stereo Correspondence ($D = 10$)}
\label{tab:counts}
\begin{tabular}{llrrrr}
\toprule
\textbf{Pair} & \textbf{Det.} & \textbf{NMS-L} & \textbf{NMS-R}
              & \textbf{Corr} & \textbf{Corr\%} \\
\midrule
1  & Harris & 12288 & 12288 & 11493 & 93.5 \\
   & SIFT   &  1964 &  2401 &   855 & 43.5 \\
   & ORB    &  1579 &  1540 &   431 & 27.3 \\
   & KAZE   &  2783 &  3053 &  1439 & 51.7 \\
\midrule
7  & Harris & 12288 & 12288 & 11595 & 94.4 \\
   & SIFT   &  2923 &  3134 &  1662 & 56.9 \\
   & ORB    &  1652 &  1674 &   516 & 31.2 \\
   & KAZE   &  2855 &  3097 &  1373 & 48.1 \\
\midrule
13 & Harris & 12288 & 12288 & 11307 & 92.0 \\
   & SIFT   &  2288 &  2138 &   929 & 40.6 \\
   & ORB    &  1428 &  1476 &   436 & 30.5 \\
   & KAZE   &  2772 &  2803 &  1481 & 53.4 \\
\bottomrule
\end{tabular}
\end{table}

Harris consistently fills every NMS grid cell (12,288 = $128 \times 96$),
indicating that its threshold is not selective enough to distinguish strong
from weak corners — the detector responds to nearly every textured pixel.
This inflates the correspondence rate to 92--94\%, since matching candidates
are virtually everywhere in the image.

KAZE achieves the best balance among the selective detectors: approximately
2,800 features per image with left-right counts within 10\%, and an average
correspondence rate of 51\%. SIFT is similar in feature count (1,964--2,923)
with correspondence averaging 47\%. ORB produces the fewest useful features
after NMS (1,428--1,652) and the lowest correspondence (27--31\%).

Figure~\ref{fig:harris_strip} illustrates the effect of varying $D$ on
feature count. Doubling $D$ from 10 to 20 reduces features by approximately
$4\times$ (as expected for a 2D grid where cell area quadruples), confirming
the geometric relationship between suppression radius and retained count.

\begin{figure}[h]
    \centering
    \includegraphics[width=\columnwidth]{images/pair1_SIFT_strip.jpg}
    \caption{Pair 1 — SIFT keypoints on rectified left (green) and right
    (orange) images with horizontal epipolar lines. Corresponding features
    lie on the same scanline, confirming successful rectification.}
    \label{fig:sift_strip}
\end{figure}

\begin{figure}[h]
    \centering
    \includegraphics[width=\columnwidth]{images/pair1_KAZE_strip.jpg}
    \caption{Pair 1 — KAZE keypoints on rectified stereo pair. KAZE produces
    well-distributed features with strong localisation at object boundaries.}
    \label{fig:kaze_strip}
\end{figure}

\begin{figure}[h]
    \centering
    \includegraphics[width=\columnwidth]{images/pair1_ALL_detectors.jpg}
    \caption{Pair 1 left image with all four detectors overlaid:
    Harris (green), SIFT (orange), ORB (pink), KAZE (cyan).
    Harris covers the image uniformly; the three selective detectors
    cluster at high-gradient object boundaries.}
    \label{fig:all_detectors}
\end{figure}

\subsection{Inter-Detector Overlap}

Table~\ref{tab:overlap} reports the percentage of keypoints from detector A
that have a spatial match within 10 pixels in detector B's output, for Pair 1.
Results for Pairs 7 and 13 are consistent.

\begin{table}[h]
\centering
\caption{Detector Overlap on Pair 1 (radius = 10 px). Entry $(A \to B)$
is the \% of A's keypoints found within 10 px of a B keypoint.}
\label{tab:overlap}
\begin{tabular}{lcccc}
\toprule
 & \textbf{Harris} & \textbf{SIFT} & \textbf{ORB} & \textbf{KAZE} \\
\midrule
Harris & ---   & 35.2 & 21.7 & 40.1 \\
SIFT   & 99.9  & ---  & 53.8 & 75.8 \\
ORB    & 99.9  & 90.7 & ---  & 93.8 \\
KAZE   & 99.6  & 63.9 & 56.8 & ---  \\
\bottomrule
\end{tabular}
\end{table}

The asymmetry is the most informative result. SIFT $\to$ Harris, ORB $\to$
Harris, and KAZE $\to$ Harris are all $\approx 99\%$: virtually every
feature found by the selective detectors is also found by Harris, confirming
Harris as a superset detector. Conversely, Harris $\to$ SIFT is only
35--45\%: Harris detects roughly 2--3$\times$ more locations than SIFT
considers strong enough to retain, meaning the extra Harris features are
mostly weak or repeated texture responses that would hamper matching.

SIFT $\to$ KAZE overlap of 75.8\% and KAZE $\to$ SIFT of 63.9\% indicate
substantial but not complete agreement between these two methods. They share
a common set of highly salient scene features while each finding additional
locations the other rejects.

\subsection{Effect of NMS Radius}

\begin{figure}[h]
    \centering
    \includegraphics[width=\columnwidth]{images/pair7_Harris_strip.jpg}
    \caption{Pair 7 — Harris response after NMS at $D = 10$. The
    uniform spatial distribution is a consequence of grid-based NMS
    filling every $10\times10$ pixel cell.}
    \label{fig:harris_strip}
\end{figure}

\begin{figure}[h]
    \centering
    \includegraphics[width=\columnwidth]{images/pair7_ORB_strip.jpg}
    \caption{Pair 7 — ORB keypoints. ORB tends to cluster features at
    high-contrast edges and avoids smooth regions, resulting in uneven
    spatial distribution and lower stereo correspondence.}
    \label{fig:orb_strip}
\end{figure}

Increasing $D$ from 10 to 20 reduces post-NMS counts by approximately
$4\times$ for Harris (12,288 $\to$ 3,072) and $2\times$ for selective
detectors (e.g., SIFT: 1,964 $\to$ 1,207 on Pair 1). The difference arises
because Harris fills every cell regardless of $D$, whereas selective detectors
leave many cells empty at $D = 10$ already. For a $1280 \times 960$ image,
$D = 10$ corresponds to cells covering $0.78\%$ of the image width, which
is consistent with the recommendation to scale $D$ with resolution
\cite{mikolajczyk2005}.

% ============================================================================
% CREATIVE COMPONENT
% ============================================================================
\section{Creative Component: Cross-Pair Repeatability}

\subsection{Motivation and Definition}

Repeatability, as defined by Mikolajczyk and Schmid \cite{mikolajczyk2005},
measures how consistently a detector fires on the same physical scene
points across different viewpoints. Formally, given a reference set of
keypoints $\mathcal{K}_{\text{ref}}$ from view 1 and a set
$\mathcal{K}_{\text{other}}$ from view 2:
\begin{equation}
    \rho(r) = \frac{\bigl|\{p \in \mathcal{K}_{\text{ref}} :
    \exists\, q \in \mathcal{K}_{\text{other}},\;
    \|p - q\|_2 < r\}\bigr|}{|\mathcal{K}_{\text{ref}}|} \times 100\%
\end{equation}
where $r$ is a spatial tolerance in pixels. This is a purely geometric
measure — no descriptor matching is performed — isolating the localization
stability of the detector itself.

\subsection{Experimental Protocol}

Pair 1 was used as the reference. Repeatability was computed against Pairs 7
and 13 at three radii $r \in \{10, 15, 20\}$ pixels. Harris was excluded
from meaningful comparison because its grid saturation produces non-selective
keypoints at every cell regardless of scene content.

\subsection{Results}

\begin{table}[h]
\centering
\caption{Cross-Pair Repeatability (\%). Pair 1 as reference.
Harris omitted due to grid saturation artifact.}
\label{tab:repeat}
\begin{tabular}{lcccccc}
\toprule
 & \multicolumn{3}{c}{\textbf{Pair 1 $\to$ Pair 7}}
 & \multicolumn{3}{c}{\textbf{Pair 1 $\to$ Pair 13}} \\
\cmidrule(lr){2-4}\cmidrule(lr){5-7}
\textbf{Det.} & $r$=10 & $r$=15 & $r$=20
              & $r$=10 & $r$=15 & $r$=20 \\
\midrule
SIFT & 52.4 & 68.0 & 77.6 & 41.8 & 67.3 & 80.5 \\
ORB  & 37.7 & 58.6 & 69.3 & 30.8 & 50.3 & 68.1 \\
KAZE & 63.8 & 73.7 & 79.5 & 62.7 & 75.2 & 78.9 \\
\bottomrule
\end{tabular}
\end{table}

\subsection{Discussion}

KAZE achieves the highest repeatability at all radii and across both pair
comparisons, reaching 73--75\% at $r = 15$ pixels. This is consistent with
the nonlinear diffusion design: because diffusion halts at image edges, KAZE
features are anchored to stable physical boundaries (corners of objects,
shelf edges) rather than texture patches that vary with illumination or
viewpoint.

SIFT achieves 67--68\% at $r = 15$ px, which closely matches the 60--70\%
range reported by Mikolajczyk and Schmid \cite{mikolajczyk2005} on their
standard viewpoint-change benchmark (bark, wall, and bikes sequences).
This agreement suggests our lab scene provides a comparable level of
viewpoint challenge to the published benchmark.

ORB achieves 50--59\% at $r = 15$ px, notably lower than SIFT and KAZE.
While ORB's FAST detector is highly reproducible under pure translation,
its sensitivity to in-plane rotation and scale change (despite the
orientation assignment) reduces repeatability when the camera position
changes substantially.

Figure~\ref{fig:repeatability} visualizes how repeatability grows with
radius for all three detectors on both pair comparisons. The curve shape
— steep growth from 10 to 15 px then flattening toward 20 px — indicates
that most repeatable features reappear within 15 pixels of their reference
location, suggesting that sub-15 px localization precision is achievable
for all methods.

\begin{figure}[h]
    \centering
    \includegraphics[width=\columnwidth]{images/pair13_KAZE_strip.jpg}
    \caption{Pair 13 — KAZE keypoints illustrating strong edge-anchored
    feature localisation that contributes to high cross-pair repeatability.}
    \label{fig:repeatability}
\end{figure}

% ============================================================================
% DISCUSSION
% ============================================================================
\section{Discussion}

\subsection{Suitability for Stereo Matching}

The results suggest a clear ranking for the next project (feature matching
and 3-D reconstruction): KAZE $>$ SIFT $>$ ORB, with Harris unsuitable
in its current form. KAZE's combination of high feature count, 51\%
stereo correspondence, and 73--75\% cross-pair repeatability makes it
the strongest candidate for producing reliable stereo matches. SIFT
provides slightly fewer features but achieves comparable repeatability
and substantially better theoretical matching properties (real-valued
descriptors vs. binary). ORB's 27--31\% stereo correspondence is a
significant weakness — roughly two thirds of ORB features in the left
image have no candidate in the right image, which would result in many
failed matches.

\subsection{Harris as a Baseline}

Harris is useful as a baseline to confirm that the NMS implementation
is working correctly (uniform coverage at $D = 10$) and to establish an
upper bound on feature density. Its 92--94\% stereo correspondence confirms
that the rectification is accurate: the images are well-aligned horizontally,
and the low $y$-tolerance criterion ($\leq 3$ pixels) is satisfied for nearly
all candidate pairs. However, Harris should be combined with a strong
descriptor and aggressive NMS before use in matching.

\subsection{Left-Right Balance}

Assignment requirements specify roughly equal feature counts in left and
right views. For Harris the counts are identical by construction (same image
resolution, same NMS grid). For SIFT and KAZE, left and right counts differ
by 10--22\%, attributed to viewpoint-dependent texture visibility —
features behind objects are visible from one camera but occluded in the
other. ORB is most balanced (within 5--10\%) due to its binary threshold
and scale pyramid design.

% ============================================================================
% CONCLUSION
% ============================================================================
\section{Conclusion}

This paper evaluated Harris, SIFT, ORB, and KAZE detectors on calibrated
rectified stereo images under five quantitative criteria. KAZE emerged as
the best-performing selective detector, combining the highest feature yield
among quality detectors (${\sim}2{,}800$ post-NMS), the best stereo
correspondence (51\%), and the highest cross-pair repeatability (73--75\%
at 15 px), attributable to its nonlinear diffusion scale space that anchors
features to stable object boundaries. SIFT achieved comparable repeatability
(67--68\%) consistent with published benchmarks \cite{mikolajczyk2005}.
ORB, while computationally fastest, produced the lowest correspondence and
repeatability, limiting its utility for dense stereo reconstruction. Harris
confirmed accurate rectification geometry through near-perfect correspondence
rates but is too non-selective for quality matching without further filtering.
These results provide a principled basis for selecting detectors and
suppression parameters in the subsequent feature matching and 3-D
reconstruction project.
% ============================================================================
% REFERENCES
% ============================================================================
\newpage
\bibliographystyle{IEEEtran}
\bibliography{references}
\balance
\end{document}